%%
\chapter*{Wstęp}
%%
\begin{quotation}
\indent Gry komputerowe już od wielu lat kreują nowe trendy. Istnieją różne rodzaje gier.
Gry typu AAA tworzone przez wielkie studia gier jak np. Ubisoft, Electronic Arts, czy też nasz rodzimy CD Project Red. 
Patrząc na gry komputerowe klasy AAA możemy śmiało powiedzieć, że przypominają filmy rodem z Disney'a czy też Pixar'a. 
W dzisiejszych czasach, aby oszczędzić sobie czasu, wiele dużych studiów używa technologii Motion Capture, aby odwzorować idealnie ruch postaci. Sprawia to,
że potrzeba manualnego rigowania postaci --- ustawienia ruchu postaci w sposób realistyczny --- przestaje być kłopotem grafików, oraz animatorów, którzy mogą skupić się na innych aspektach swoich zadań.

\indent Gry typu Indie to produkcje, które zazwyczaj są tworzone przez jedną osobę, bądź amatorskie studio złożone z kilku osób. Ze względu na brak funduszy oraz wyposażenia studia, produkcje te zazwyczaj są niskiej jakości, bądź średni czas potrzebny do przejścia gry jest o wiele krótszy niż gry AAA.
W dzisiejszych czasach istnieje dużo (pełno to kolokwializm) gier, których grafika składa się z pikseli, voxeli czyli sześciokątów oraz w stylu low-poly który zostanie wykorzystany do tego projektu.

\newpage 
\end{quotation}

%%
