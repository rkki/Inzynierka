%%
\chapter*{Wstęp}
%%
\begin{quotation}
\indent Gry komputerowe już od wielu lat kreują nowe trendy. Istnieją różne rodzaje gier.
Gry typu AAA tworzone przez wielkie studia gier jak np. Ubisoft, Electronic Arts, czy też nasz rodzimy CD Project Red. 
Patrząc na gry komputerowe klasy AAA możemy śmiało powiedzieć, że przypominają filmy rodem z Disney'a czy też Pixar'a. 
W dzisiejszych czasach, aby oszczędzić sobie czasu, wiele dużych studiów używa technologii Motion Capture, aby odwzorować idealnie ruch postaci. Sprawia to,
<<<<<<< HEAD
że potrzeba manualnego animowania postaci przestaje być kłopotem grafików, oraz animatorów, którzy mogą skupić się na innych aspektach swoich zadań.

\indent Gry typu Indie to produkcje, które zazwyczaj są tworzone przez jedną osobę, bądź amatorskie studio złożone z kilku osób. Ze względu na brak funduszy oraz wyposażenia studia, produkcje te zazwyczaj są niskiej jakości, bądź średni czas potrzebny do przejścia gry jest o wiele krótszy niż gry AAA. Już dekadę temu, wielu graczy zmieniło upodobania co do grafiki, jaka pojawia się w grach komputerowych. Nawiązania do gier z przed dekady uwarunkowane są sentymentem do tamtego okresu. W 2019 do nagrody Paszportu Polityki 2019 została nominowana gra Dawida Ciślaka "We. The Revolution" ~która ukazuje nasze spory wokół sądów i polityki, jednakże podczas czasów rewolucji francuskiej. W dzisiejszych czasach istnieje dużo gier, których grafika składa się z pikseli, voxeli czyli sześciokątów oraz w stylu low-poly który zostanie wykorzystany do tego projektu.

\indent Low-poly jest to angielski termin określający siatkę modelu 3D, która składa się z małej liczby poligonów. Twórcy gier uważają, że low-poly stało się prekursorem grafiki gier komputerowych u dużej ilości osób rozpoczynających swoją przygodę z tworzeniem gier komputerowych. Wykonanie każdego obiektu jest o wiele prostsze od modeli 3D przedstawiających realistycznie postać człowieka bądź drzewa. Ten styl graficzny posiada też swój urok, który można nazwać stylem kreskówkowym. Obiekty są wyraźnie zaznaczone i posiadają cechy charakterystyczne, które pozwalają odróżnić je od innych elementów rozgrywki.
=======
że potrzeba manualnego rigowania postaci --- ustawienia ruchu postaci w sposób realistyczny --- przestaje być kłopotem grafików, oraz animatorów, którzy mogą skupić się na innych aspektach swoich zadań.

\indent Gry typu Indie to produkcje, które zazwyczaj są tworzone przez jedną osobę, bądź amatorskie studio złożone z kilku osób. Ze względu na brak funduszy oraz wyposażenia studia, produkcje te zazwyczaj są niskiej jakości, bądź średni czas potrzebny do przejścia gry jest o wiele krótszy niż gry AAA.
W dzisiejszych czasach istnieje dużo (pełno to kolokwializm) gier, których grafika składa się z pikseli, voxeli czyli sześciokątów oraz w stylu low-poly który zostanie wykorzystany do tego projektu.
>>>>>>> 50ed26ecdf0053dd6ea854a744027c8c69c43bb6

\newpage 
\end{quotation}

%%
