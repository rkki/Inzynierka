%%
\chapter{Wstęp}
%%
\begin{quotation}
\indent Pierwsze komputery powstały z myślą o obliczeniach balistycznych, wytwarzaniem broni jądrowej, metrologii oraz badaniu promieniowania kosmicznego. W latach 70. XX wieku, gry komputerowe były produkowane w masowych ilościach ze względu na wzrost popularności automatów do gier oraz pierwszych konsol. 30 lat później, twórcy podzespołów komputerowych produkują sprzęt z uwagą na to, aby najnowsze produkcje gier działały w jak najlepszej rozdzielczości, z jak nawiększą liczbą klatek na sekundę. Prawdziwie prestiżowym wynikiem jest granie w rozdzielczości 4K w 60 klatkach na sekundę.

\indent Gry komputerowe już od wielu lat kreują nowe trendy, przez co powstało wiele różnych rodzajów gier. Gry typu AAA tworzone przez wielkie studia gier jak np. Ubisoft, Electronic Arts, czy też nasz rodzimy CD Project RED. Patrząc na gry komputerowe klasy AAA możemy śmiało powiedzieć, że przypominają filmy rodem z Disney'a czy też Pixar'a. W dzisiejszych czasach, aby oszczędzić sobie czasu, wiele dużych studiów używa technologii Motion Capture, aby odwzorować idealnie ruch postaci. Sprawia to, że potrzeba manualnego animowania postaci przestaje być kłopotem grafików, oraz animatorów, którzy mogą skupić się na innych aspektach swoich zadań.

\indent Gry typu Indie to produkcje, które zazwyczaj są tworzone przez jedną osobę, bądź amatorskie studio złożone z kilku osób. Ze względu na brak funduszy oraz wyposażenia studia, produkcje te zazwyczaj są niskiej jakości, bądź średni czas potrzebny do przejścia gry jest o wiele krótszy niż gry AAA. Obecnie, wielu graczy zmieniło upodobania co do grafiki, jaka pojawia się w grach komputerowych. \\
Nawiązania do gier sprzed dekady uwarunkowane są sentymentem do tamtego okresu. Gry wtedy były robione z pasji. Aktualnie pojawia się wiele skandali co do wielkich firm tworzących gry wyłącznie dla pieniędzy. Przykładem może być Electronic Arts, który 8 marca 2012r wydał grę Mass Effect 3, a dzień później udostępnili DLC, czyli zawartość dodatkową za opłatą wynoszącą około 50\% ceny gry. Gracze byli oburzeni, dlatego zaczęli masowo zwracać gry do sklepu. Ponieważ liczba zwrotów wynosiła blisko 60\% zakupionych wersji gry, EA zdecydowało udostępnić tylko ten dodatek za darmo, następne były już płatne.

\indent W 2019 roku do nagrody ,,Paszportu Polityki 2019'' została nominowana gra Dawida Ciślaka "We. The Revolution" ~która ukazuje nasze spory wokół sądów i polityki, które dzieją się podczas czasów rewolucji francuskiej. 18 lipca 2019r ukazała się gra Horace stworzona przez niezależne studio 505 Games. Została ona stworzona w stylu pixelowym, z naciskiem na grafikę 2D, która w niektórych scenach zmienia się na 3D. Poprzez takie wymieszanie rodzajów grafik, mieszankę styli oraz interesującą historią otrzymała przydomek ,,Najlepszej platformówki 2019 roku''. W dzisiejszych czasach istnieje dużo gier, których grafika składa się z pikseli, voxeli czyli sześciokątów oraz w stylu low-poly, który zostanie wykorzystany do projektu opisanego w tej pracy.

\indent Low-poly jest to angielski termin określający siatkę modelu 3D, która składa się z małej liczby poligonów. Twórcy gier uważają, że low-poly stało się standardowym stylem grafiki gier komputerowych u dużej ilości osób rozpoczynających swoją przygodę z tworzeniem gier. Wykonanie każdego obiektu jest o wiele prostsze od modeli 3D przedstawiających realistycznie postać człowieka bądź drzewa. Ten styl graficzny posiada też swój urok, który można nazwać stylem kreskówkowym. Obiekty są wyraźnie zaznaczone i posiadają cechy charakterystyczne, które pozwalają odróżnić je od innych elementów rozgrywki.
\end{quotation}