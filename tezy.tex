%%
\chapter{Cel pracy, wymagania użytkownika, założenia, zakres i~tezy pracy}
%%
\section{Cel pracy}\label{cel}
%%
Celem pracy jest stworzenie projektu gry komputerowej za pomocą silnika Unity której grafika będzie reprezentowała styl graficzny ,,low-poly$"$.
%%

%%
\section{Wymagania użytkownika}
%%
Przez użytkownika (zamawiającego) zdefiniowane zostały następujące wymagania:
\begin{itemize}
\item gra musi być stworzona w środowisku Unity,
\item projekt musi być przedstawiony w formie 3D,
\item gra powinna posiadać świat przedstawiony w postaci low-poly.
\end{itemize}
%%
%%
%%
\section{Założenia}
%%
Przy realizacji projektu przyjęto następujące założenia:
\begin{itemize}
\item do projektu należy stworzyć modele 3D samochodu oraz przeszkód,
\item do projektu należy zaprogramować skrypty obsługujące sterowanie pojazdem oraz mechanikę związaną z przechodzeniem poziomów gry.
\end{itemize}
%%
%%
%%
\section{Zakres}
%%
Zakres pracy obejmował:
\begin{itemize}
\item elementy modeli ze względu na braki funduszy, zostały stworzone w programie Blender,
\item pracę ograniczono jedynie do pięciu krótkich poziomów, ze względu na aktualny stan projektu oraz komplikacje związane z rozmiarem.
\end{itemize}
%%
%%
%%
\section{Teza pracy}
%%
Na podstawie przeglądu literatury dotyczącego zagadnienia przyjęto następujące tezy pracy:
\begin{itemize}
  \item do prawidłowego zrealizowania projektu wystarczające jest środowisko Unity, Blender oraz.
\end{itemize}
%%