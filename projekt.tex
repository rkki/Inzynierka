%%
\chapter{Badania własne}
\begin{quotation}
\indent Proces badawczy to proces planowego, celowego wykorzystania uzyskanych informacji oraz własnej wiedzy. Celem głównym pracy jest próba stworzenia projektu świata przedstawionego w stylu graficznym low-poly. Głównym problemem jest rozmiar gry i złożoność poziomów. Zadecydowano więc, że projekt będzie przedstawiał pięć poziomów, w których gracz musi omijać przeszkody aby przejść na poziom wyżej.

\indent Przeprowadzając badania własne, postanowiono zapoznać się z rynkiem narzędzi służących do tworzenia grafiki 3D. Wybrano 3DS Max, Autodesk Maya, Cinema 4D, ZBrush, Blender oraz Houdini.

\indent Przeszukując możliwe silniki i oprogramowania do tworzenia gier, zdecydowano zaznajomić się z programami RPG Maker, GameMaker, Unreal Engine oraz Unity.
%%
\section{Program do modelowania 3D}

\indent Rynek programów służących do modelowania 3D jest nastawiony na specyficzne projekty. Każde oprogramowanie wybierane jest pod projekt i najczęściej są to programy, które posiadają reputację niezniszczalnych programów, które są niezawodne. Jeżeli firma zajmuje się projektem, który posiada nieprzekraczalny termin dostarczenia, pewnym jest, że wybierze program, który posiada wsparcie konsumenta. Jest to jedna z głównych przyczyn, dlaczego takie programy jak Blender, które są wolnym i otwartym oprogramowaniem, nie są wybierane, ponieważ nie posiadają personelu zajmującego się awariami. Każdy artysta ma swój ulubiony program. Często jest tak, że kierownik projektu z góry narzuca oprogramowanie. Obecnie, do modelowania 3D, animowania, efektów specjalnych oraz rzeźbienia jest używane przynajmniej 5 programów, które nie każdy może znać. Wywiera to presję na artyście. Nie może on skupić się na pracy, ponieważ musi nauczyć się obsługi oprogramowania, które prawdopodobnie zmieni się przy kolejnym projekcie. Wymóg znajomości większości programów do modelowania 3D stał się priorytetem podczas zatrudniania nowego personelu.

\indent 3DS Max jest znany już od dawna. Pierwsza wersja programu ukazała się w 1996r pod nazwą ,,3D Studio Max 1.0''. Posiada bardzo silną pozycję na rynku i jest jednym z najczęściej wybieranych programów do tworzenia grafiki 3D. Program jest przystosowany do tworzenia animacji postaci, ze względu na zaimplementowany system, Character Studio. To oprogramowanie jest często używane w firmach zajmujących się filmami, animacjami oraz grami.

\indent Pierwsza wersja Autodesk Maya zadebiutowała w 1998r. Jeżeli grafikowi zależy na efektach specjalnych, Maya jest najbardziej profesjonalnym wyborem. Posiada solidny zestaw narzędzi, który jest nieobecny w niektórych programach. Niestety profesjonalizm idzie w parze z kosztami. Maya jest jednym z programów graficznych, który posiada najbardziej wygórowaną cenę, 200£ miesięcznie. Firmy takie jak Pixar, korzystają praktycznie tylko z Maya, bowiem ich produkcje wymagają jak najlepszych efektów. Jest to jeden z programów który jest wychwalany przez artystów oraz grafików, ponieważ posiada niesamowity wachlarz narzędzi. Poprzez ilość opcji, funkcji oraz mechanizmów jest również bardzo skomplikowany, przez co jego nauka jest bardzo trudna. Duża liczba osób zaczynających przygodę z modelowaniem 3D, często sięga po darmową wersję trial, jednakże po pierwszym spotkaniu z interfejsem jak i trudnościami w stworzeniu czegokolwiek, szybko zostawiają to oprogramowanie w spokoju.

\newpage
\indent Jeżeli artyście zależy tylko na animacjach, idealnym oprogramowaniem będzie Cinema 4D. Oprogramowanie to, jest prawdopodobnie jednym z najstarszych na tej liście, bowiem pierwsza jego wersja ukazała się w 1990r. Prawdopodobnym jest, że w większości filmów zawierających nienaturalne animacje, których aktor nie dał rady wykonać, znajduje się animacja stworzona w Cinema 4D. 

\indent ZBrush jest programem który jest nastawiony na Sculpting, czyli rzeźbienie. Tworząc zwykłą kulę a następnie modyfikując ją wszelakimi pędzlami, które dodają, bądź odejmują geometrię, jesteśmy w stanie uzyskać rzeźby identyczne jak w rzeczywistości. Jest to idealny program do tworzenia nieruchomych postaci, które przykładowo pozują. Najczęściej jest wykorzystywany do stworzenia obiektu przypominającego jakiegoś aktora lub postać, który następnie jest użyty w reklamie jako statyczna postać. 

\indent Houdini w głównej mierze używany jest do tworzenia obiektów proceduralnych. Są to obiekty, które są tworzone przez funkcje i modyfikatory, które przy różnych ustawieniach wartości mogą ukazać inny wynik. Jest on tańszym wariantem Autodesk Maya, ponieważ graficy używają go do tworzenia animacji na obiektach wcześniej stworzonych w innym programie do tworzenia grafiki 3D. Często Houdini jest też wykorzystywany do animacji obiektów. Niestety nie jest on na tyle dobrą alternatywą dla Autodesk Maya.

\indent Blender z kolei, jest połączeniem każdego ww. oprogramowania. Posiada praktycznie identyczne funkcje 3DS Maxa pod względem modyfikowania siatki meshowej obiektów 3D. Jeżeli chcemy uzyskać delikatne efekty, można spokojnie użyć do tego Blendera, który poradzi sobie dobrze z naszym problemem, chociaż w Maya udałoby się uzyskać ten sam efekt o wiele lepiej. Animacje w Blenderze polegają jedynie na wyobraźni autora. Korzystając z wszelakich transformów oraz modyfikatorów, animacje nie stanowią problemu. Rzeźbienie w Blenderze, przy ustawieniach podstawowych jest trudniejsze niż w ZBrush'u. Ponieważ istnieje opcja importowania gotowych pędzli, dopiero po zebraniu dużej ilości narzędzi przystosowanych do jednego celu, jesteśmy w stanie uzyskać praktycznie ten sam wachlarz narzędzi co w ZBrushu. W przypadku tworzenia obiektów proceduralnych, Blender radzi sobie bardzo dobrze. Aby uzyskać efekt jaki możemy dostać w Houdini czy Maya, animator musi się solidnie napracować, co nie oznacza, że jest to niemożliwe do osiągnięcia.

\indent Do realizacji pracy wybrano Blendera, ponieważ każdy z elementów rozgrywki jaki ukaże się w grze jest możliwy do wykonania w tym programie. Ponieważ obiekty 3D potrzebne do projektu nie są wymagające, a efekty specjalne jak i animacje obiektów są nieobecne, do wykonania modeli wykorzystano program Blender. 

\section{Styl grafiki -- Low-poly}

\indent Pierwsza gra, która posiadała ,,prawdziwe'' 3D, Descent ukazała się w 1995r i została stworzona przez wydawcę gier Interplay Entertainment. Grafikę skonstruowano z poligonów i jako jedna z niewielu gier tamtych czasów pozwalała na sześć stopni swobody, czyli możliwość poruszania się postacią w każdą możliwą stronę horyzontalnie i wertykalnie.

\indent Graficy, tworząc obiekty 3D używają sformułowań odpowiednich do tworzonego przez nich kształtu. Trójkąty popularnie nazywane trisami oraz prostokąty, czyli poligony, posiadają odpowiednią dla figury geometrycznej liczbę ścian i vertexów. \textbf{Vertexy}, są to punkty łączenia krawędzi poligonu. Istnieją również n-gony, czyli figury składające się z liczby ścian większej niż 4. W przypadku, kiedy obiekt nie jest animowany, nie ma żadnych przeciwskazań. Należy jednak pamiętać, że w przypadku, gdy obiekt musi być animowany, powinno się unikać takich rozwiązań, ponieważ silniki renderowania grafiki, mają problem z obliczeniem odbicia światła od powierzchni zawierającej więcej niż 4 vertexy.

\indent Oprawa graficzna w grach jest jednym z najważniejszych kryteriów, która wpływa na to, czy ktoś będzie grał w tą produkcję. Gracze są jednym z najgorszych celów marketingowych. Każdy ma inne wymagania oraz upodobania, przez co trudno trafić w większość. Dzisiejsze produkcje twórców zajmujących się produkcjami AAA opierają się na fotorealistyczności. Jak można się domyślić, nie każdemu to pasuje. Spora liczba graczy preferuje gry, które posiadają wygląd przypominający kreskówki, bądź komiksy. Jedną z takich gier jest ,,The Wolf Among Us'' producenta Telltale Games z 2013r. Gra przedstawiona jest w postaci 3D, a styl graficzny jest połączeniem fotorealistycznych modeli z kreskówkowymi elementami. Przedstawia ona świat, w którym są umieszczone postacie z najbardziej znanych baśni, które często ukazują zachowania patologiczne. Główne postacie żyją w Nowym Jorku, jednej z największych metropolii świata.

\indent W przypadku styli gier komputerowych określających wymiary przestrzeni, możliwym jest wymienienie głównych przedstawicieli. Gry w stylu 3D zachowujące fotorealizm przodują w rankingach popularności, dlatego też największe studia takie jak Electronic Arts czy Ubisoft skupiają się właśnie na fotorealiźmie. Pomniejsze firmy, których produkty nie są rozpoznawalne na całym globie mają inne zdanie. Każdy ich projekt jest innym przedsięwzięciem celowanym w zupełnie inną publikę. Tacy producenci szukają swojego stylu. W przypadku jeżeli ich gra spodoba się sporej liczbie graczy, próbują stworzyć inną grę, która swoją oprawą graficzną będzie przypominać poprzednią produkcję. Nie zmienia to faktu, że duża liczba gier jest ukazana w środowisku trójwymiarowym. Istnieją też zawzięci fanatycy konkretnych rodzai gier.

\indent Gry 2D przeżywają swego rodzaju renesans. Takie gry jak Horace, Stardew Valley, Darkest Dungeon itp. podbijają serca graczy swoją grafiką i historią. Większość z nich posiada bardzo skomplikowane mechaniki, na które 3D nie pozwala, bądź wymagany jest bardzo duży wkład czasowy. Mechaniki te mają na celu pokazanie odbiorcy, że nie tylko grafika ma ich cieszyć a mechaniki związne z rozgrywką.

\indent Gry dzielą się w głównej mierze na 3D i 2D. Ale nie każda gra jest równa drugiej. Dochodzą przeróżne style graficzne. Możliwym jest wylistowanie kilku z najpopularniejszych. Cartooning, czyli po prostu komiks, jest mniej popularny niż reszta styli, co nie zmienia faktu, że posiada swój urok. Świetnie wykonane postacie i charakterystyczna czarna kreska tworząca kontur postaci, kolory są bardzo wyraźne.

\indent Flat design, czyli płaskie elementy zawierające kształty geometryczne. Jak każdy ze styli, posiada fanów oraz krytyków. Spora liczba gier na telefony jest stworzona w stylu flat design. Jego całkowitym przeciwieństwem jest styl fotorealistyczny 3D. Popularnie uważanym jest, że ten styl zawarty jest tylko w poważnych grach. Zaletą jest poziom detali otoczenia jak i postaci. Minusem jest czas pracy nad poszczególnymi elementami, co również czyni ten styl bardzo kosztownym.

\indent Low-poly jest określeniem siatki modelu 3D, który charakteryzuje się małą ilością wielokątów popularnie zwanych poligonami. Poligony są wykorzystywane do tworzenia obiektów 3D. Siatka modelu tworzona jest poprzez łączenie poligonów w jedną spójną całość. Na przestrzeni lat, patrząc na początki gier komputerowych, dzisiejsze miano low-poly róźnie się diametralnie od swoich poprzedników z przed dwóch dekad. Ze względu na o wiele większą ilość poligonów oraz technologię dzięki której tworzenie modeli 3D jest o wiele łatwiejsze, ten styl dalej odbiega od modeli które są nastawione na realizm. Poligony teoretycznie mogą posiadać nieskończoną ilość boków, jednakże silniki renderowania grafiki często napotykają problem z poligonami, których ilość boków jest większą od pięciu, zwłaszcza przy animacjach ukazujących obiekt w ruchu. Nieoficjalnie ustalono, że siatka modelu musi składać się z trójkątów, bądź prostokątów. Aby obiekt w stylu low-poly posiadał detale prawdziwego obiektu, używa się map normalnych i map wypukłości, które cieniują poligony i dodają im dodatkowej geometrii. Mapy pomagają uwydatnić detale, których ze względu na ilość poligonów w siatce obiektu, nie da się uwzględnić ręcznie.

\indent Rozwój technologii silników gier umożliwił nowe rozwiązania dla obiektów low-poly. Odkąd silniki oraz technologia obliczeniowa procesorów rozwinęła się w wielkim stopniu, obecne obliczenia są w stanie przedstawić setki tysięcy poligonów w standardowych 25 klatkach. 
W produkcjach AAA low-poly dalej istnieje, jednakże zostało ograniczone do wykorzystywania mniej detalicznych modeli jako forma zmniejszenia jakości oddalonych obiektów. LOD czyli Level Of Detail, definiuje zasięg z jakim dany obiekt posiada jakość detali. Im dalej znajduje się obiekt, tym gorsze odwzorowanie są detali, które nie pobierają tak dużo mocy obliczeniowej w przypadku głównej sceny, przez co gra działą płynnie \cite{1}.

\indent W przypadku kiedy twórca zdecycował, że chce użyć stylu low-poly, nie ma powodu, żeby używać LODa, ponieważ modele 3D w tym stylu zawierają wystarczającą liczbę detali jak i nie wymagają wielkiej mocy obliczeniowej ze względu na małą liczbę krawędzi.

\begin{figure}[hbt!]
\setcaptionwidth{0.75\linewidth}
\centering
  \includegraphics[width=0.75\linewidth]{lowpolytuto.jpg}
  \caption{Przykładowy render wyspy stworzonej w stylu low-poly}\label{rys_1}
  \begin{minipage}[t]{0.75\linewidth}
    \emph{Źródło: YouTube, CG Geek, Low Poly Island | Beginner | Blender 2.8 Tutorial, 2019.}
  \end{minipage}
\end{figure}


\section{Środowisko tworzenia gry}

\indent Tworząc grę, pierwszą decyzją musi być historia, szata graficzna a następnie wybór silnika oraz środowiska w którym będzie tworzona gra. Ponieważ rynek gier przez ostatnie dwie dekady powiększył się, niektóre silniki postanowiły otworzyć się na świat i stały się wolnymi od opłat oprogramowaniami. Silnikiem określa się zbiór gotowych funkcji potrzebnych do stworzenia gry, które często posiadają zintegrowane środowisko. Silnik zajmuje się komunikowaniem pomiędzy poszczególnymi elementami projektu.

\indent Istnieje wiele programów oraz silników do tworzenia gier komputerowych. Wiodące w branży gier studia, takie jak Ubisoft, Electronic Arts, CD Projekt RED korzystają z własnych silników. Osoby, które tworzą gry w pojedynkę, korzystają z gotowych silników, bądź próbują pisać własne. Najpopularniejszymi gotowymi silnikami są Unity oraz Unreal Engine.

\indent RPG Maker wydany przez firmę Degica jest jednym z programów, który pozwala tworzyć proste gry komputerowe, nie wymagając znajomości języka programowania. Jeżeli twórca zapoznał się z programem i nie chce go zmieniać, musi nauczyć się JavaScript aby tworzyć gry o bardziej skomplikowanych mechanikach.

\indent W 1998r zadebiutowała gra Unreal wyprodukowana przez Epic MegaGames i Digital Extremes, która powstała na pierwszej wersji silnika Unreal Engine I. W 2006r ukazała się trzecia wersja silnika Unreal do której licencję wykupiło wiele przedsiębiorstw takich jak Electronic Arts, Activision czy też Ubisoft, które dziś są jednymi z najbardziej szanowanych studiów zajmujących się tworzeniem gier. W 2015r Unreal Engine stał się darmowy dla każdego, kto chciał rozpocząć przygodę z tworzeniem gier. Spora liczba twórców wybiera Unreal Engine ze względu na możliwość kodowania na klockach, które służą do kodowania wizualnego. \\
Ten moduł wykorzystuje się głównie do prototypowania gier, co nie zmienia faktu, że można tak stworzyć całą grę nie pisząc kodu.

\indent Podobnym do RPG Maker jest następny program, GameMaker stworzona przez firmę YoYo Games. Pierwsze wydanie ukazało się w 1999r. Jest to środowisko do tworzenia gier i programów komputerowych, w którym nie jest wymagana znajomość języka programowania. Ponieważ program jest przeznaczony dla osób nie znających się na programowaniu, GameManager posiada wiele gotowych funkcji, przystosowanych pod konkretne zachowania rozgrywki.  Tak jak Unreal, posiada opcję tworzenia skomplikowanych rozwiązań poprzez łączenie funkcji wizualnie. Środowisko posiada wbudowany język skryptowy GML, tj. GameMaker Language, który porzypomina JavaScript. Został on stworzony z naciskiem na tworzenie gier \cite{2}.

\indent Unity ukazało się w 2005r i początkowo był wykorzystywany w filmach, architekturze i symulacjach. W 2015r trafił do darmowej dystrybucji i szybko stało się konkurencją dla Unreal Engine. Dodatkowo Unity posiada Unity Asset Store, czyli sklep z gotowymi modelami, skryptami a nawet całymi projektami gier. Inspektor, czyli okno akcji każdego obiektu jak i całego projektu, pozwala modyfikować zmienne i dodawać nowe komponenty. Zaimportowany obiekt można dowolnie zmieniać, tj. dodać materiał, kiedy np. importowany obiekt był tylko siatką meshową. W przypadku kiedy w skrypcie zakodowano zmienną, której nadano przykładową wartość, twórca podczas testowania jest w stanie zmieniać jej wartość nie modyfikując kodu, również podczas procesu testowania, właśnie w Inspektorze. Do pisania kodu, wybrano Microsoft Visual Studio 2017, ponieważ darmowa wersja programu jest dołączana do pakietu instalacyjnego Unity. Aktualnie jedynym językiem programowania jest C\#, ponieważ silnik Unity posiada gotowe metody i fukcje, które współgrają wyłącznie z tym językiem. 

\newpage
Unity posiada największą z ww. programów ilość platform do eksportowania gry, konsole takie jak XBOX, PlayStation, komputery stacjonarnych z Windowsem, Mac'iem czy Linuxem a także telefony z iOS bądź Androidem. Do realizacji projektu wybrano Unity, ponieważ pozwala on na stworzenie takiej samej gry jak w Unreal, przy mniejszym wkładzie czasowym. 

\end{quotation}
%%