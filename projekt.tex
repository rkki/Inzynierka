%%
\chapter{Badania własne}
\indent Proces badawczy to proces planowego, celowego wykorzystania uzyskanych informacji oraz własnej wiedzy. Celem głównym pracy jest próba stworzenia projektu świata przedstawionego w stylu graficznym low-poly. Głównym problemem jest rozmiar gry i złożoność poziomów. Zadecydowano więc, że projekt będzie przedstawiał pięć poziomów, w których gracz musi omijać przeszkody aby przejść na poziom wyżej.

\indent W 1998r zadebiutowała gra Unreal wyprodukowana przez Epic MegaGames i Digital Extremes, która powstała na pierwszej wersji silnika Unreal Engine I. Swoją relatywnie prostszą strukturą wygryzła silnik Quake II Engine, który został wykorzystany w grze Quake. W 2006 ukazała się trzecia wersja silnika Unreal do której licencję wykupiło wiele przedsiębiorstw takich jak Electronic Arts, Activision czy też Ubisoft, które dziś są jednymi z najbardziej szanowanych studii zajmujących się tworzeniem gier. W 2015 roku, Unreal Engine stał się darmowy dla każdego, kto chciałby rozpocząć przygodę z tworzeniem gier.

\indent Unity ukazało się w 2005r i początkowo był wykorzystywany w filmach, architekturze i symulacjach. W 2015r trafiło do darmowej dystrybucji i szybko stało się konkurencją dla Unreal. Dodatkowo Unity posiada Unity Asset Store, czyli sklep z gotowymi modelami, skryptami a nawet całymi projektami gier.

\indent Nawzajem z każdą aktualizacją, obydwa silniki były rozwijane w taki sposób, aby udowodnić że są lepsze i mają więcej możliwości.
%%
\section{Blender}
\indent Rozpoczynając pracę związaną z tworzeniem grafiki 3D, postanowiłem dowiedzieć się o możliwych programach do właśnie takich zadań. Z pośród najbardziej popularnych można wymienić Blendera, Autodesk Maya, Houdini oraz ZBrush. Jednym z kryteriów wyboru była cena zakupu programu. Oczywistym wyborem jest Blender, ponieważ jest on darmowym oprogramowaniem oraz posiada większość opcji z ww. programów. Modele wykorzystane w projekcie muszą być stworzone w stylu low-poly, oznacza to, że wymagania co do złożoności siatki obiektu pod względem topologii nie są wygórowane, wręcz muszą być zaniżone. 
%%
\section{Styl grafiki -- Low-poly}

\indent Low-poly jest określeniem siatki modelu 3D, który charakteryzuje się małą ilością wielokątów popularnie zwanych poligonami. Na przestrzeni lat, patrząc na początki gier komputerowych, dzisiejsze miano low-poly róźnie się diametralnie od swoich poprzedników z przed dwóch dekad, ze względu na o wiele większą ilość poligonów oraz technologię dzięki której tworzenie modeli 3D jest o wiele łatwiejsze, jednakże dalej odbiega od modeli które są nastawione na realizm. Wówczas obiekt zachowuje znajomą nam geometrię. Poligony teoretycznie mogą posiadać nieskończoną ilość boków, jednakże silniki renderowania grafiki często napotykają problem z ilością większą od pięciu, zwłaszcza przy animacjach. Nieoficjalnie ustalono, że siatka modelu musi składać się z trójkątów, bądź prostokątów. Aby obiekt w stylu low-poly posiadał detale prawdziwego obiektu, używa się map normalnych i map wypukłości, które cieniują poligony i dodają jej dodatkowej geometrii, aby posiadały detale, których ze względu na rozmiar siatki obiektu, nie da się uwzględnić ręcznie. 

\indent Rozwój technologii silników przedstawił nowe rozwiązania dla obiektów low-poly. Odkąd silniki oraz technologia obliczeniowa procesorów rozwinęła się w wielkim stopniu, obecne obliczenia są w stanie przedstawić setki tysięcy poligonów w standardowych 25 klatkach. W produkcjach AAA low-poly zostało ograniczone do jakości oddalonych obiektów. LOD czyli Level Of Detail, definiuje zasięg z jakim dany obiekt posiada jakość detali. Im dalej znajduje się obiekt, tym gorsze odwzorowanie detali, jednakże nie pobiera tak dużo mocy obliczeniowej w przypadku głównej sceny w pobliżu postaci głównej. \cite{1}

\indent W przypadku kiedy twórca zdecycował, że chce użyć stylu low-poly, nie ma powodu, żeby używać LODa, ponieważ modele 3D w stylu low-poly zawierają wystarczającą liczbę detali jak i nie wymagają wielkiej mocy mocy obliczeniowej.


%%

\begin{figure}[hbt!]
\setcaptionwidth{0.75\linewidth}
\centering
  \includegraphics[width=0.75\linewidth]{lowpolytuto.jpg}
  \caption{Przykładowy render wyspy stworzonej w stylu low-poly}\label{rys_1}
  \begin{minipage}[t]{0.75\linewidth}
    \emph{Źródło: YouTube, CG Geek, Low Poly Island | Beginner | Blender 2.8 Tutorial, 2019.}
  \end{minipage}
\end{figure}

%%
\section{Unity}

\indent Istnieje wiele programów oraz silników do tworzenia gier komputerowych. Wiele dużych studiów takich jak Ubisoft, EA, CDProjektRED korzystają z własnych silników które sami napisali. Osoby, które tworzą gry w pojedynkę, korzystają z gotowych silników. Najpopularniejszymi są Unity oraz Unreal Engine. Wybierając Unity kierowałem się własnymi doświadczeniami związanymi z grami komputerowymi i programowaniem. Dużo gier, które miałem okazję zobaczyć było tworzone właśnie na silniku Unity. Silnik pozwala programować w dwóch językach skryptów, C\# który jest podobny do C++, oraz JavaScript. C\# jest dla mnie bardzo intuicyjny ze względu na wielkie podobieństwo do C++ z którym miałem już doczynienia. 

\indent Inspektor, czyli okno akcji każdego obiektu jak i całego projektu, pozwala modyfikować zmienne i dodawać nowe komponenty. Zaimportowany obiekt można dowolnie zmieniać, tj. dodać materiał, kiedy np. importowany obiekt był tylko siatką meshową. W przypadku kiedy w skrypcie zakodowano zmienną, której nadano przykładową wartość, twórca podczas testowania jest w stanie zmieniać jej wartość nie modyfikując kodu, również podczas procesu testowania, właśnie w Inspektorze. 

%%