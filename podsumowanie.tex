%%
\newpage
\chapter{Podsumowanie}
\begin{quotation}

\indent Proces tworzenia gier komputerowych wymaga od twórcy poświęcenia czasu na zapoznanie się z historią stylu graficznego oraz jego wymaganiami. Tworząc grę low-poly należy pamiętać, aby modele 3D posiadały siatkę modelu która będzie wykonana poprawnie. Ważnym elementem tworzenia gier w konkretnym stylu jest zebranie informacji dotyczących wymagań topologii modeli. Charakterystyczne cechy stylu low-poly to siatka modelu składająca się z trisów i poligonów prostokątnych. Obiekt nie może posiadać dużej ilości poligonów, ponieważ jak sama nazwa wzkazuje, nie będzie to low-poly. Małe detale trzeba ograniczyć do minimum, bądź całkowicie wykluczyć a kolory obiektów muszą się wyróżniać, aby modele nie nakładały się na siebie. 

\indent Tworząc obiekt, którego pierwowzór ma odzwierciedlenie w rzeczywistości, należy pamiętać aby stworzyć go w taki sposób, aby jak najbardziej przypominał tą rzecz, chociażby pod względem kształtu. Jedno z narzędzi Blendera, Extrude, jest najlepszym przyjacielem twórcy. Za jego pomocą jesteśmy w stanie dowolnie rozszerzać siatkę modelu dodając przy tym dodatkowej objętości. Najczęstszym sposobem na modyfikowanie obiektu jest jego skalowanie. Modyfikując poszczególne elementy obiektu, twórca jest w stanie uzyskać nowe krawędzie, które nadadzą modelowi wymaganej geometrii. Modyfikator Mirror, jest przydatny w momentach, kiedy twórca skupia się na jednej stronie modelu, aby przesuwając i modyfikując konkretną część obiektu, uzyskać idealne odbicie lustrzane. W przypadku gdy twórca chce uzyskać ciekawy efekt wizualny obiektu, zamiast tworzyć ręcznie trisy na siatce obiektu, może użyć modyfikatora Triangulate, który zmodyfikuje siatkę meshową modelu, przerabiając każdy możliwy poligon na dwa trisy. Niestety wiąże się to z konsekwencją zwiększenia ilości poligonów w obiekcie, co może doprowadzić do komplikacji związanych z przewidywaną liczbą ścian w każdym obiekcie.

\indent Umiejętność operowania w danym programie, który służy się do tworzenia gier, również jest przydatna, ze względu na czas. Tworząc obiekty w Blenderze, po wcześniejszym nauczeniu się programu, skrótów oraz dostępnych modyfikatorów, możliwym jest stworzenie modelu w krótkim czasie. Korzystając z silnika gry, który wymaga pisania skryptów, aby uzyskać przeróżne mechaniki rozgrywki, dodatkowo rozwijana jest znajomość języka programowania, użyteczna w następnych projektach. Otrzymanie żądanego efektu może zająć sporo czasu, jednakże podczas powstawania kolejnego projektu, jesteśmy w stanie uzyskać ten sam efekt, bądź lepszy, w o wiele krótszym czasie. Blender jako darmowe oprogramowanie często jest pomijany przez wielkie firmy zajmujące się animacjami oraz modelowaniem 3D. Jest to spowodowane brakiem wsparcia technicznego, które jest wymagane w przypadku, kiedy podczas procesu tworzenia ważnego projektu, wyłączy się program i nie można go uruchomić ponownie, bądź wszystkie dane zostały stracone. Nie zmienia to faktu, że Blender dorównuje swoim konkurentom i prędzej czy później osiągnie status jednego z najlepszych oprogramowań dostępnych za darmo na rynku.

\indent Jak widać w rozdziale \textbf{3.4}, aby stworzyć grę wymagana jest podstawowa wiedza o języku programowania, oraz znajomość funkcji silnika Unity. Wszystkie gotowe funkcje i metody zawarte w silniku Unity są udostępnione w Unity User Manual dostępnym na stronie głównej Unity. Podczas pisania kodu ważnym jest aby zwracać uwagę na pisanie pierwszych liter. C\# jest językiem programowania, który zwraca uwagę na to, czy funkcja jest napisana z dużej litery czy nie. Standardem jest, aby zmienne zaczynały się z małej litery a metody i fukcje z dużej. \\
Istnieją też dwie szkoły pisania długich nazw. Aby oddzielić od siebie słowa korzysta się ze snake'a albo camel'a. \textbf{to\_jest\_snake} czyli oddzielanie słów za pomocą znaku podkreślenia, a \textbf{toJestCamel}, czyli pisanie pierwszego słowa z małej litery, a następnie zaczynanie każdego następnego słowa z wielkiej. 
\indent Należy pamiętać o konieczności przystosowania gry do każdego komputera. Do metod które wpływają na rozgrywkę, takie jak poruszanie się, należy dodać \textit{Time.deltaTime} aby sprzęt na którym się gra, nie dawał przewagi. Poprzez \textit{ForceMode.VelocityChange} twórca jest w stanie uzyskać element prawdziwego poruszana się dwóch różnych obiektów o tej samej prędkości. Dodatkowo ten tryb pozwala uzyskać sposób poruszania się obiektów po scenie w stylu gier typu Arcade.

\indent Spełnione zostały wymagania użytkownika, a teza została udowodniona. Gra została stworzona w środowisku Unity w formie 3D, a modele przeszkód i pojazdu zostały wykonane w postaci low-poly. Plik instalacyjny gry został stworzony w programie Inno Setup a następnie nagrany na płytę. Założenia projektu obejmowały stworzenie modeli 3D oraz zaprogramowanie skryptów obsługujących sterowanie i mechaniki gry. Powstały trzy modele przeszkód oraz jeden pojazd. Podczas procesu programowania powstało 6 skryptów zajmujących się rozgrywką, oraz dwa obsługujące przyciski startu oraz wyjścia z gry. 
\end{quotation}
%%