%%
\chapter{Podsumowanie}

\indent Podsumowując, proces tworzenia gier komputerowych wymaga poświęcenia czasu na zapoznanie się z historią oraz wymaganiami co do stylu graficznego. Tworząc grę low-poly należy pamiętać, aby modele 3D posiadały siatkę modelu nieodbiegającą od resztu. 

\indent Umiejętność operowania w danym programie, który przysłużył się do stworzenia gry, również jest przydatna, ze względu na czas. Tworząc obiekty w Blenderze, po wcześniejszym nauczeniu się programu, możliwym jest stworzenie porządnego obiektu w krótkim czasie. Znajomość jednego programu pozwala szybko nauczyć się drugiego, przykładowo Autodesk Maya. 

\indent Wynikiem pracy inżynierskiej jest gra, która została nagrana na płytę DVD. Została ona dołączona do pracy.
%%