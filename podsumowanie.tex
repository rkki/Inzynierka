%%
\chapter{Podsumowanie}

\indent Podsumowując, proces tworzenia gier komputerowych wymaga poświęcenia czasu na zapoznanie się z historią stylu graficznego oraz jego wymaganiami. Tworząc grę low-poly należy pamiętać, aby modele 3D posiadały siatkę modelu która będzie wykonana poprawnie. Ważnym elementem tworzenia gier w konkretnym stylu jest zebranie informacji dotyczących wymagań topologii modeli. Charakterystyczne cechy danego stylu jak i typowe dla niego mechaniki muszą spełniać wymagania użytkowników, a rozgrywka musi być przyjemna oraz przyjazna dla oka.

\indent Umiejętność operowania w danym programie, który przysłużył się do stworzenia gry, również jest przydatna, ze względu na czas. Tworząc obiekty w Blenderze, po wcześniejszym nauczeniu się programu, możliwym jest stworzenie porządnego obiektu w krótkim czasie. Korzystając z silnika gry, który wymaga pisania skryptów, aby pokazać przeróżne mechaniki rozgrywki, dodatkowo rozwija znajomość języka programowania. 

\indent Spełnione zostały wymagania użytkownika. Gra została stworzona w środowisku Unity w formie 3D, a modele środowiska, przeszkód i pojazdu zostały wykonane w postaci low-poly. Założenia projektu obejmowały stworzenie modeli 3D oraz zaprogramowanie skryptów obsługujących sterowanie i mechaniki gry. 

\indent Wynikiem pracy inżynierskiej jest gra, która jest nagrana na płytę DVD. Została ona dołączona do pracy.
%%