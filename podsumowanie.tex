%%
\chapter{Podsumowanie}
\begin{quotation}
\indent Podsumowując, proces tworzenia gier komputerowych wymaga poświęcenia czasu na zapoznanie się z historią stylu graficznego oraz jego wymaganiami. Tworząc grę low-poly należy pamiętać, aby modele 3D posiadały siatkę modelu która będzie wykonana poprawnie. Ważnym elementem tworzenia gier w konkretnym stylu jest zebranie informacji dotyczących wymagań topologii modeli. Charakterystyczne cechy stylu low-poly to siatka modelu składająca się z trisów i poligonów prostokątnych. Obiekt nie może posiadać dużej ilości poligonów, ponieważ jak sama nazwa wzkazuje, nie będzie to low-poly. Małe detale trzeba ograniczyć do minimum, bądź całkowicie wykluczyć. Kolorystyka obiektów musi być żywa, dobrym zwyczajem jest używanie naturalnych kontrastów.

\indent W Blenderze tworząc obiekt zawierający skomplikowaną topologię, oraz posiadający dużą liczbę detali, możemy użyć modyfikatora obiektu ,,Decimate''. W przypadku, kiedy siatka obiektu posiada ponad tysiąc poligonów, możliwym jest ograniczenie ilości poligonów właśnie poprzez tą funkcję. Decimate redukuje liczbę vertexów i ścian siatki meshowej przy jak najmniejszych zmianach kształtu obiektu, co pozwala uzyskać efekt low-poly. Metoda ta jest najczęściej używana w przypadku kiedy obiekt został stworzony poprzez rzeźbienie obiektu a liczba poligonów w siatce obiektu wynosi dziesiątki, bądź setki tysięcy. 

\indent Umiejętność operowania w danym programie, który przysłużył się do stworzenia gry, również jest przydatna, ze względu na czas. Tworząc obiekty w Blenderze, po wcześniejszym nauczeniu się programów, skrótów oraz modyfikacji, możliwym jest stworzenie porządnego obiektu w krótkim czasie. Korzystając z silnika gry, który wymaga pisania skryptów, aby uzyskać przeróżne mechaniki rozgrywki, dodatkowo rozwijana jest znajomość języka programowania, która pomaga w następnych projektach. Uzyskanie żądanego efektu może zająć sporo czasu, jednakże podczas następnego projektu, jesteśmy w stanie uzyskać ten sam efekt, bądź lepszy w o wiele krótszym czasie.

\indent Spełnione zostały wymagania użytkownika. Gra została stworzona w środowisku Unity w formie 3D, a modele środowiska, przeszkód i pojazdu zostały wykonane w postaci low-poly. Założenia projektu obejmowały stworzenie modeli 3D oraz zaprogramowanie skryptów obsługujących sterowanie i mechaniki gry. 

\indent Wynikiem pracy inżynierskiej jest gra, która jest nagrana na płytę DVD. Została ona dołączona do pracy.
\end{quotation}
%%